\documentclass[8pt, letterpaper, titlepage]{article}
\usepackage[utf8]{inputenc}
\usepackage{geometry}
\usepackage{color,graphicx,overpic} 
\usepackage{fancyhdr}
\usepackage{amsmath,amsthm,amsfonts,amssymb}
\usepackage{mathtools}
\usepackage{hyperref}
\usepackage{multicol}
\usepackage{array}
\usepackage{float}
\usepackage{blindtext}
\usepackage{longtable}
\usepackage{scrextend}
\usepackage[font=small,labelfont=bf]{caption}
\usepackage[framemethod=tikz]{mdframed}
\usepackage{calc}
\usepackage{titlesec}
\usepackage{listings}
\usepackage[normalem]{ulem}
\usepackage{tabularx}
\usepackage{mathrsfs}
\usepackage{bookmark}
\usepackage{setspace}
\usepackage{tabularx}
\usepackage{ltablex}
\usepackage{enumitem}
\usepackage[simplified]{pgf
-umlcd}
\definecolor{dkgreen}{rgb}{0,0.6,0}
\definecolor{gray}{rgb}{0.5,0.5,0.5}
\definecolor{mauve}{rgb}{0.58,0,0.82}
\usepackage{listings}

\lstset{frame=tb,
  language=Java,
  aboveskip=3mm,
  belowskip=3mm,
  showstringspaces=false,
  columns=flexible,
  basicstyle={\small\ttfamily},
  numbers=none,
  numberstyle=\tiny\color{gray},
  keywordstyle=\color{blue},
  commentstyle=\color{dkgreen},
  stringstyle=\color{mauve},
  breaklines=false,
  breakatwhitespace=true,
  tabsize=3
}

\mathtoolsset{showonlyrefs}  
\allowdisplaybreaks

\definecolor{mycolor}{rgb}{0, 0, 0}

\geometry{top=2.54cm, left=2.54cm, right=2.54cm, bottom=2.54cm}

% Indentation/space between paragraphs
\setlength{\headheight}{15pt}
\setlength{\parindent}{0pt}
\setlength{\parskip}{0pt}

% Line spacing
\renewcommand{\baselinestretch}{1.5} 

% Line spacing
\renewcommand{\baselinestretch}{1.3} 

% Title page
\title{\textbf{\Huge{ 
\begin{center}
MATE 201\\ \large{Class notes} % Document name
\end{center} 
}}}

\author{Lora Ma}

% Header/Footer
\pagestyle{fancy}
\fancyhf{}
\rhead{\thepage}
\lhead{\textit{CMPUT 325 Quiz (Part of Assignment 2)}}
\rfoot{}

% Hyperlink colors
\hypersetup{
    colorlinks=true,
    linkcolor=blue,
    filecolor=blue,      
    urlcolor=blue,
}

\begin{document}

\section*{Part A}

\subsection*{Q1.}
Normal order reduction:
\begin{align}
    &(\lambda xyz \mid xz(yyz)) (\lambda x \mid x) (\lambda x \mid xy)\ a \\
    &\rightarrow (\lambda x \mid(\lambda yz \mid xz(yyz))) (\lambda x \mid x) (\lambda x \mid xy)\ a \\
    &\rightarrow (\lambda x \mid (\lambda y \mid(\lambda z \mid xz(yyz)))) (\lambda x \mid x) (\lambda x \mid xy)\ a \\
    &\rightarrow (\lambda y \mid(\lambda z \mid (\lambda x \mid x)z(yyz))) (\lambda x \mid xy)\ a \\
    &\rightarrow (\lambda z \mid (\lambda x \mid x)z((\lambda x \mid xy)(\lambda x \mid xy)z))\ a \\
    &\rightarrow (\lambda x \mid x)a((\lambda x \mid xy)(\lambda x \mid xy)a) \\
    &\rightarrow a((\lambda x \mid xy)(\lambda x \mid xy)a) \\
    &\rightarrow a(((\lambda x \mid xy)y)a) \\
    &\rightarrow a(yya) \\
\end{align}

Applicative order reduction:
\begin{align}
    &(\lambda xyz \mid xz(yyz)) (\lambda x \mid x) (\lambda x \mid xy)\ a \\
    &\rightarrow (\lambda x \mid (\lambda yz \mid xz(yyz))) (\lambda x \mid x) (\lambda x \mid xy)\ a \\
    &\rightarrow  (\lambda yz \mid (\lambda x \mid x)z(yyz)) (\lambda x \mid xy)\ a \\
    &\rightarrow (\lambda yz \mid z(yyz)) (\lambda x \mid xy)\ a \\
    &\rightarrow (\lambda y \mid (\lambda z \mid z(yyz))) (\lambda x \mid xy)\ a \\
    &\rightarrow (\lambda z \mid z((\lambda x \mid xy)(\lambda x \mid xy)z)) a \\
    &\rightarrow (\lambda z \mid z(((\lambda x \mid xy)y)z)) a \\
    &\rightarrow (\lambda z \mid z((yy)z)) a \\
    &\rightarrow a(yya) \\
\end{align}

\subsection*{Q2.}
\begin{enumerate}[label=(\alph*)]
  \item Knowing XOR has the following logic
  \begin{lstlisting}
    XOR (a, b):
      if a:
        return not(b)
      else:
        return b
  \end{lstlisting}
  we can combine this with the lambda expression given for NOT and OR to obtain
  \begin{align}
    &x \text{ XOR } y \\
    &\rightarrow (\lambda xy \mid x(\text{NOT } y) y)\\
    &\rightarrow (\lambda xy \mid x(\lambda y \mid y FT) y)\\
    &\rightarrow (\lambda xy \mid x(y FT) y)\\
  \end{align}
  \item \begin{itemize}
    \item For $\text{XOR } TF$,
    \begin{align}
      &(\lambda xy \mid x(y FT) y)\\
      &\rightarrow T(F FT) F \\
      &\rightarrow T(T)F \\
      &\rightarrow T
    \end{align}
    \item For $\text{XOR } TT$,
    \begin{align}
      &(\lambda xy \mid x(y FT) y)\\
      &\rightarrow T(T FT) T \\
      &\rightarrow T(F)T \\
      &\rightarrow F
    \end{align}
  \end{itemize}
\end{enumerate}
\subsection*{Q3.}
\begin{enumerate}[label=(\alph*)]
  \item The result of evaluating the expression is 13 and the last context created during evaluation is $\{x \rightarrow 5, y \rightarrow 3\} \cup$ CT0
    \begin{enumerate}[label=(\roman*)]
      \item 
      Evaluate (lambda (x) (lambda (y) (+ (* 2 x) y))) 5
      \begin{itemize}
        \item CT1 = $\{x \rightarrow 5\} \cup $ CT0
        \item Then, [F1, CT1] = [lambda (y) (+ (* 2 x) y), CT1] where F1 = (lambda (y) (+ (* 2 x) y))
      \end{itemize}
      \item Evaluate outer expression with [F1, CT1]: (F1 3)
      \begin{itemize}
        \item CT2 = $\{y \rightarrow 3\} \cup $ CT1 = $\{x \rightarrow 5, y \rightarrow 3\} \cup$ CT0
        \item Then, [F2, CT2] = [(+ (* 2 x) y), CT2] where F2 = (+ (* 2 x) y)
      \end{itemize}
      \item Evaluate [F2, CT2]: (F2)
      \begin{itemize}
        \item F2 evaluates to (+ (* 2 x) y) with CT2 = $\{x \rightarrow 5, y \rightarrow 3\} \cup$ CT0
      \end{itemize}
      \item Finally, 
      \begin{itemize}
        \item (+ (* 2 x) y) = (+ (* 2 5) 3) = 13
      \end{itemize}
    \end{enumerate}
    \item The result of evaluating the expression is 9 and the last context created during evaluation is $\{x \rightarrow 5, y \rightarrow 3\} \cup$ CT0
    \begin{enumerate}[label=(\roman*)]
      \item Evaluate (lambda (x) (+ 1 x)) 
      \begin{itemize}
        \item CT1 = CT0
        \item Then, [F1, CT1] = [(+ 1 x), CT1] where F1 = (+ 1 x)
      \end{itemize}
      \item Evaluate the second argument 8
      \item Evaluate outer expression [F1, CT1] and 8: (F1 8)
      \begin{itemize}
        \item CT2 = $\{x \rightarrow 8\} \cup $ CT1 = $\{x \rightarrow 8\} \cup $ CT0 
        \item Then,[F2, CT2] = [(+ 1 x), CT2] where F2 = (+ 1 x)
      \end{itemize}
      \item Evaluate [F2, CT2]: (F2)
      \begin{itemize}
        \item F2 evaluates to (+ 1 x) with CT2 = $\{x \rightarrow 8\} \cup $ CT0
      \end{itemize}
      \item Finally, 
      \begin{itemize}
        \item (+ 1 x) = (+ 1 8) = 9
      \end{itemize}
    \end{enumerate}
\end{enumerate}
\end{document}