\documentclass[8pt, letterpaper, titlepage]{article}
\usepackage[utf8]{inputenc}
\usepackage{geometry}
\usepackage{color,graphicx,overpic} 
\usepackage{fancyhdr}
\usepackage{amsmath,amsthm,amsfonts,amssymb}
\usepackage{mathtools}
\usepackage{hyperref}
\usepackage{multicol}
\usepackage{array}
\usepackage{float}
\usepackage{blindtext}
\usepackage{longtable}
\usepackage{scrextend}
\usepackage[font=small,labelfont=bf]{caption}
\usepackage[framemethod=tikz]{mdframed}
\usepackage{calc}
\usepackage{titlesec}
\usepackage{listings}
\usepackage[normalem]{ulem}
\usepackage{tabularx}
\usepackage{mathrsfs}
\usepackage{bookmark}
\usepackage{setspace}
\usepackage{tabularx}
\usepackage{ltablex}
\usepackage{enumitem}
\usepackage[simplified]{pgf
-umlcd}
\definecolor{dkgreen}{rgb}{0,0.6,0}
\definecolor{gray}{rgb}{0.5,0.5,0.5}
\definecolor{mauve}{rgb}{0.58,0,0.82}
\usepackage{listings}

\lstset{frame=tb,
  language=Java,
  aboveskip=3mm,
  belowskip=3mm,
  showstringspaces=false,
  columns=flexible,
  basicstyle={\small\ttfamily},
  numbers=none,
  numberstyle=\tiny\color{gray},
  keywordstyle=\color{blue},
  commentstyle=\color{dkgreen},
  stringstyle=\color{mauve},
  breaklines=false,
  breakatwhitespace=true,
  tabsize=3
}

\mathtoolsset{showonlyrefs}  
\allowdisplaybreaks

\definecolor{mycolor}{rgb}{0, 0, 0}

\geometry{top=2.54cm, left=2.54cm, right=2.54cm, bottom=2.54cm}

% Indentation/space between paragraphs
\setlength{\headheight}{15pt}
\setlength{\parindent}{0pt}
\setlength{\parskip}{0pt}

% Line spacing
\renewcommand{\baselinestretch}{1.5} 

% Line spacing
\renewcommand{\baselinestretch}{1.3} 

% Title page
\title{\textbf{\Huge{ 
\begin{center}
MATE 201\\ \large{Class notes} % Document name
\end{center} 
}}}

\author{Lora Ma}

% Header/Footer
\pagestyle{fancy}
\fancyhf{}
\rhead{\thepage}
\lhead{\textit{CMPUT 325 Quiz (Part of Assignment 2)}}
\rfoot{}

% Hyperlink colors
\hypersetup{
    colorlinks=true,
    linkcolor=blue,
    filecolor=blue,      
    urlcolor=blue,
}

\begin{document}

\section*{Question 1}
\subsection*{Normal order reduction}
\begin{align}
  (\lambda xyz | xz(yyz))(\lambda x | x)(\lambda x | xy) a &= (\lambda x | (\lambda y | (\lambda z | xz(yyz))))(\lambda x | x)(\lambda x | xy)a \\
  &= (\lambda y | (\lambda z | (\lambda x|x)z(yyz)))(\lambda x | xy) a \\
  &= (\lambda z | (\lambda x | x) z ((\lambda x | xy)(\lambda x | x y) z))a \\
  &= (\lambda x | x)a((\lambda x|xy)(\lambda x|xy) a) \\
  &= a((\lambda x | xy)(\lambda x | xy) a) \\
  &= a(((\lambda x | xy) )a) \\
  &= a(((y)y)a) \\
  &= a(yya)
\end{align}
\subsection*{Applicative order reduction}
\begin{align}
  (\lambda xyz | xz(yyz))(\lambda x | x)(\lambda x | xy) a &=  (\lambda yz | (\lambda x | x) z(yyz))(\lambda x | xy) a \\
  &= (\lambda yz | z(yyz))(\lambda x | xy) a \\
  &= (\lambda z | z((\lambda x | xy)(\lambda x | xy)z)) a \\
  &= (\lambda z | z(((\lambda x | xy)y)z)) a \\
  &= (\lambda z | z(((y)y)z)) a \\
  &= a(((y)y)a) \\
  &= a(yya)
\end{align}
\section*{Question 2}
\begin{enumerate}[label=(\alph*)]
  \item \begin{align}
    x \text{ OR } y &= (\lambda xy | x\text{T}y) \\
    \text{NOT }x &= (\lambda x | x \text{FT}) \\
    % x \text{ AND } y &= (\lambda x y | xy \text{F}) \\
    % \text{ NOT } (x \text{ AND } y) &= (\lambda x | (\lambda x y | xy \text{F}) \text{FT}) \\
  \end{align}
We can intuitively determine the logic behind XOR as \lstinline|xor(x, y) = {x ? not y : y}|. Knowing this, we can formulate the following lambda expression:
\begin{align}
  x \text{ XOR } y &= (\lambda x y | (x (\lambda y | y \text{FT})y)) \\
  &= (\lambda x y | (x (y \text{FT})y)) \\
  &= (\lambda x y |x (y \text{FT})y) \\
\end{align} 
\item 
\begin{itemize}
  \item \begin{align}
    \text{XOR T F} &= (\lambda x y |x (y \text{FT})y) \\
    &= \text{T} (\text{F} \text{FT})\text{F} \\
    &= \text{TTF} \\
    &= \text{T}
  \end{align}
  \item \begin{align}
    \text{XOR T T} &= (\lambda x y |x (y \text{FT})y) \\
    &= \text{T} (\text{T} \text{FT})\text{T} \\
    &= \text{TFT} \\
    &= \text{F}
  \end{align}
\end{itemize}
\end{enumerate}
\section*{Question 3}
\begin{enumerate}[label=(\alph*)]
  \item \lstinline|(((lambda (x) (lambda (y) (+ (* 2 x) y))) 5) 3)|
  \begin{itemize}
    \item \lstinline|(((lambda (x) (lambda (y) (+ (* 2 x) y))) 5) 3)|
    \item \lstinline|((lambda (x) (lambda (y) (+ (* 2 x) y))) 5)|
    \item \lstinline|(lambda (x) (lambda (y) (+ (* 2 x) y)))| 
    \item \lstinline|(lambda (y) (+ (* 2 x) y))| \\
    CT1 = $[x \rightarrow 5] \cup$ CT0 \\
    $[e, \text{CT1}]$ = [\lstinline|(lambda (y) (+ (* 2 x) y))|, CT1] 
    \item \lstinline|(+ (* 2 x) y)|  \\
    CT2 = $[x \rightarrow 5] \cup$ $[y \rightarrow 3] \cup$ CT0 $= [x \rightarrow 5, y \rightarrow 3]$ \\
    $[e1, \text{CT2}]$ = [\lstinline|(+ (* 2 x) y)|, CT2] 
    \item \lstinline|(+ (* 2 x) y)|  
    \item \lstinline|(+ (* 2 5) 3)| = 13
  \end{itemize}
  \textbf{Result:} 13 \\
  \textbf{Last context: } $\{x \rightarrow 5, y \rightarrow 3\} \cup$ CT0 \\
  \item \lstinline|((lambda (x y) (x y)) (lambda (x) (+ 1 x)) 8)| \\
  \begin{itemize}
    \item \lstinline|(lambda (x) (+ 1 x))| in [$x \rightarrow$ \lstinline|(lambda (x) (+ 1 x))|, $y \rightarrow 8$] $\cup$ CT0
    CT1 = CT0 \\
    $[e, \text{CT1}]$ = [\lstinline|(+ 1 x)|, CT1] 
    \item \lstinline|(lambda (x y) (x y))| \\
    $[e1, \{\}]$ = [\lstinline|(+ (* 2 x) y)|, \{\}] =  [lambda (x) (+ 1 x), {x → [lambda (x) (+ 1 x), \{\}]}]
    \item \lstinline|(+ 1 x)| with CT2 = $\{x \rightarrow 8\} \cup$ CT0
    \item \lstinline|(+ 1 8)| = 9
  \end{itemize}
  \textbf{Result:} 9 \\
  \textbf{Last context: } \{$x \rightarrow 8$\} $\cup$ CT0
\end{enumerate}
\end{document}